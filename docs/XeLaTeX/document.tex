\documentclass[UTF8]{article} %  using ctex
\title{Document of GiveYouAHead}
\author{Qinka\\qinka@live.com}

\usepackage{xeCJK}
\setCJKmainfont{SimSun}

\usepackage{tabularx}

\usepackage{color,titlesec,titletoc,listings}
\usepackage[colorlinks,linkcolor=blue,anchorcolor=blue,citecolor=red,bookmarksnumbered]{hyperref} %lstlisting settings
%换行
\lstset{breaklines}
%背景色
\lstset{backgroundcolor=\color[rgb]{0.84,1.00,0.92}{}}
%格式
\lstset{basicstyle=\sffamily,keywordstyle=\bfseries,commentstyle=\rmfamily\itshape,escapechar='}
%非等宽
\lstset{flexiblecolumns}
%行号设置
\lstset{numbers=left,numberstyle=\tiny}
%针对汉字的逃逸字符
\lstset{escapechar=`}

\lstset{frame=trBL}
%标号
\lstset{label=sourceCtr}

\renewcommand\thesection{\Roman{section}}

% section 
\titleformat{\section}[frame]{\bfseries\Huge}{\thesection}{1em}{}[]
\titlecontents{section}[1em]{\bfseries\Large}{\thesection}{}{\titlerule*[1pt]{.}\contentspage}

\begin{document}
	
\maketitle
\newpage
\tableofcontents
\newpage
\section{About}
GiveYouAHead is a small and simple tool which is build to try helping students who are the beginners in program learning to manage their codes.
\subsection{What is this?}
This tool is just like a make-tool. You can manage your codes, and coding-homework. Teachers just need to write some templates about source, build, clean, and so on. And students just need to create or do something via using the template wrote by teachers. 
\section{Source \& Install}
GiveYouAHead is an open-source tool. And you can get source from
\href{https://github.com}{GitHub} in my \href{https://github.com/Qinka/GiveYouAHead}{repo}. And at the same time, this tool is distributed by source via hackage{http://hackage/haskell.org}.

I will also distribute this tool by the binary. There will be Windows 64bit version, linux 64bit version, linux 32bit version\footnote{This version might be build under the ghc-7.6.x, but others will be build under the ghc-7.10.x}, OS X version \footnote{This will be build at os x 10.11, and the rootless is enabled.}.
You can download these on my repo's release page.

\subsection{Install}
The way install this tool, is simple. You can install 
 
 
\section{The Usage of Template}
In this section, I will tell you how to write a template of GiveYouAHead.
\subsection{The General Usage}
\subsection{Writing Template for New}
When writing a template of new, there are something you have to know.
\subsubsection{The Inner-Macros of New}
The following table list the macros which is needed.
\begin{center}
    \begin{tabular}{|p{0.25\columnwidth}|p{0.4\columnwidth}|p{0.4\columnwidth}|}
    \hline \rule[-2ex]{0pt}{5.5ex} MACRO-NAME & DESCRIPTION & NOTES \\ 
    \hline \rule[-2ex]{0pt}{5.5ex} timeNow & By using this macro, you can get current time. & This macro already defined, you can just use it. \\ 
    \hline \rule[-2ex]{0pt}{5.5ex} numLeft & This is about file's name. It is a macro for the number's left part of file's name. & You must define this macro by yourself. \\ 
    \hline \rule[-2ex]{0pt}{5.5ex} numRight & This macro is same with numLeft. & You must define this macro by yourself. \\ 
    \hline \rule[-2ex]{0pt}{5.5ex} importList & This macro is about the list of the list of importing. Use this macro, and there will be the list of import.   & This macro is inner-defined. \\ 
    \hline \rule[-2ex]{0pt}{5.5ex} importLeft & This macro is the left part of import, such as "\#include<" & You have to define this macro, if you want to use it.  \\ 
    \hline \rule[-2ex]{0pt}{5.5ex} importRight & This macro is smae with importLeft, such as ">" & You have to define this macro, if you want to use it. \\ 
    \hline \rule[-2ex]{0pt}{5.5ex} E
    \hline 
\end{tabular} 

\end{center}

\end{document}


